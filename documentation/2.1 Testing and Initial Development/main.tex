% !TEX root = main.tex
\documentclass[11pt]{article}

% ---------- Basics -----------
\usepackage[margin=1in]{geometry}
\usepackage[T1]{fontenc}
\usepackage[utf8]{inputenc}
\usepackage{lmodern}
\usepackage{microtype}
\usepackage{xcolor}
\usepackage{hyperref}
\usepackage{enumitem}
\usepackage{booktabs}
\usepackage{longtable}
\usepackage{array}
\usepackage{parskip}
\usepackage{listings}
\usepackage{tabularx}

\newcolumntype{L}{>{\raggedright\arraybackslash}X}

\hypersetup{
  colorlinks=true,
  linkcolor=black,
  urlcolor=blue!60!black,
  citecolor=black
}

% ---------- Listings ----------
\lstset{
  basicstyle=\ttfamily\small,
  columns=fullflexible,
  breaklines=true,
  frame=single,
  framerule=0.2pt,
  tabsize=2,
  showstringspaces=false
}

% ---------- Title ----------
\newcommand{\product}{OpenBid}
\title{DIGT2107: Practice of Software Development\\
\Large Project Iteration 3: Testing and Initial Development\\[4pt]
\large \product}
\author{Team 1: Tyler \and Mani \and Yanness \and Alaister}
\date{Due: November 2, 2025}

\begin{document}
\maketitle

\begin{center}
\textbf{Course:} DIGT2107 -- Fall Term 2025 \quad|\quad
\textbf{Instructor:} Dr.\ May Haidar
\end{center}

\newpage

% ===================== 1. Introduction =====================
\section{Introduction}
\textbf{Project Name:} \product \\
\textbf{Team Number:} \textbf{1} \\
\textbf{Team Members:} Tyler, Mani, Yanness, Alaister

\paragraph{Document Overview}
This document outlines the deliverables for Iteration 3, focusing on the initial development phase. It shows how our \textit{test cases} map to the \textit{functional and non-functional requirements} from earlier iterations, and it reports the current working prototype. Goals: implement core functionality tied to high-priority stories, develop comprehensive unit tests, and deliver a working prototype demonstrating traceability.

% ===================== 2. Iteration Goals and Objectives =====================
\section{Iteration Goals and Objectives}
\begin{itemize}[leftmargin=1.4em]
  \item Implement core flows: \textbf{auth + Duo 2FA}, \textbf{KYC gate} (stubbed via Stripe Identity sandbox token), \textbf{post a job}, \textbf{browse on map}, \textbf{place a bid}.
  \item Develop and run unit tests mapped to requirements (frontend component tests and backend handler tests).
  \item Deliver a prototype demonstrating requirement \(\leftrightarrow\) test case links.
\end{itemize}

% ===================== 3. Requirement to Test Case Traceability =====================
\section{Requirement to Test Case Traceability}
Tables below map each requirement to the test cases that validate it. Coverage status will be updated as testing completes.

\subsection*{Functional Requirements (FR)}
\begin{longtable}{@{}p{0.12\linewidth} p{0.55\linewidth} p{0.15\linewidth} p{0.14\linewidth}@{}}
\toprule
\textbf{Req ID} & \textbf{Requirement Description} & \textbf{Test Case IDs} & \textbf{Coverage} \\
\midrule
\endhead
FR-001 & System shall allow authenticated users to \textbf{post jobs} with title, description, budget, photos, and location. & TC-001, TC-002 & Partially Covered \\
FR-002 & System shall allow providers to \textbf{browse jobs on a map} with basic filters (radius, category). & TC-003a-d, TC-004 & Partially Covered \\
FR-003 & System shall allow providers to \textbf{place bids} on open jobs; posters can view and accept a bid. & TC-005, TC-006 & Partially Covered \\
FR-004 & System shall enforce \textbf{KYC for all users} before posting or bidding. & TC-007, TC-008a--f, TC-009a--b, TC-010a--b, TC-011a--b, TC-012, TC-018a--f & Covered \\
FR-005 & System shall support \textbf{user authentication} with \textbf{Duo 2FA} for sensitive actions. & TC-008 & Partially Covered \\
FR-006 & System shall provide \textbf{in-thread messaging} per job after acceptance. & TC-009 & Planned \\
FR-007 & System shall record \textbf{ratings and reviews} after job completion. & TC-010 & Planned \\
FR-008 & System shall allow users to \textbf{manage their profile} including avatar upload and personal information. & TC-019a--d & Covered \\
\bottomrule
\end{longtable}

\subsection*{Non-Functional Requirements (NFR)}
\begin{longtable}{@{}p{0.12\linewidth} p{0.55\linewidth} p{0.15\linewidth} p{0.14\linewidth}@{}}
\toprule
\textbf{Req ID} & \textbf{Requirement Description} & \textbf{Test Case IDs} & \textbf{Coverage} \\
\midrule
\endhead
NFR-001 & \textbf{Performance:} Search/browse should return results within 800\,ms P95 for a metro with 5k open jobs (stub data). & TC-011 & Planned \\
NFR-002 & \textbf{Security:} Only KYC-verified users can hit write endpoints for jobs/bids. &  & Planned \\
NFR-003 & \textbf{Usability:} Map view and list view maintain accessible contrast and keyboard navigation for core actions. & TC-013 & Planned \\
\bottomrule
\end{longtable}

\paragraph{Notes}
\begin{itemize}[leftmargin=1.4em]
  \item Each requirement is validated by one or more test cases. Gaps are flagged as \textit{Planned}.
  \item NFR coverage status is informative only (not required for grading), but we track it to guide future work.
\end{itemize}

% ===================== 4. Detailed Test Case Descriptions =====================
\section{Detailed Test Case Descriptions}
For brevity we include the highest-priority cases now; the full catalog will live in the repo under \texttt{tests/}.

\subsection*{TC-001}
\textbf{Title:} Post Job - Valid Inputs \\
\textbf{Requirement:} FR-001 \\
\textbf{Preconditions:} User is logged in, KYC status = verified. \\
\textbf{Steps:}
\begin{enumerate}[leftmargin=1.4em]
  \item Navigate to \texttt{/new-job}.
  \item Enter valid title, description, budget, and pick a map location (mocked Google Maps).
  \item Upload a photo (mock file).
  \item Click \texttt{Post}.
\end{enumerate}
\textbf{Expected Result:} Job document is created in Firestore; UI redirects to Job Detail. \\
\textbf{Actual Result:} To be filled after execution. \\
\textbf{Status:} Pending \quad \textbf{Priority:} High

\subsection*{TC-002}
\textbf{Title:} Post Job - Validation Errors \\
\textbf{Requirement:} FR-001 \\
\textbf{Preconditions:} Logged in, KYC verified. \\
\textbf{Steps:}
\begin{enumerate}[leftmargin=1.4em]
  \item Navigate to \texttt{/new-job}.
  \item Leave title empty; click \texttt{Post}.
\end{enumerate}
\textbf{Expected Result:} Client-side validation shows error; no write occurs. \\
\textbf{Status:} Pending \quad \textbf{Priority:} High

\subsection*{TC-003a}
\begin{tabularx}{\linewidth}{|l|L|}
\hline
\textbf{Test Case ID} & TC-003a \\ \hline
\textbf{Title} & The haversineFormulakm function accurately calcuates distance between 2 coordinates. \\ \hline
\textbf{Requirement} & FR-002 \\ \hline
\textbf{Preconditions} & Valid coordinates are input. \\ \hline
\textbf{Steps} & 
\begin{minipage}[t]{\linewidth}
\begin{enumerate}[leftmargin=1.4em,nosep,topsep=0pt]
  \item Coordinates are input into the function.
  \item Function calculates the distance between the coordinates.
  \item Function returns the distance between the coordinates. \vspace{1em}
\end{enumerate}
\end{minipage} \\ \hline
\textbf{Expected Result} & Returns distance in km of the 2 coordinates, returns infinity when there is invalid input. \\ \hline
\textbf{Actual Result} & Passed - Returns distance accurately within a specific tolerance. Coordinates tested are up to 250km apart. \\ \hline
\textbf{Status} & \textcolor{green}{Passed} \\ \hline
\textbf{Priority} & High \\ \hline
\end{tabularx}

\subsection*{TC-003b}
\begin{tabularx}{\linewidth}{|l|L|}
\hline
\textbf{Test Case ID} & TC-003b \\ \hline
\textbf{Title} & The isValidCoords formula accurately determines whether the given inputs are valid latitude and longitude values. \\ \hline
\textbf{Requirement} & FR-002 \\ \hline
\textbf{Preconditions} & None. \\ \hline
\textbf{Steps} & 
\begin{minipage}[t]{\linewidth}
\begin{enumerate}[leftmargin=1.4em,nosep,topsep=0pt]
  \item Coordinates are input into the function.
  \item Function checks input for validity then checks whether the input values are valid coordinates.
  \item Function returns whether the coordinates are valid. \vspace{1em}
\end{enumerate}
\end{minipage} \\ \hline
\textbf{Expected Result} & Returns true or false to depending on whether the input is valid coordinates. \\ \hline
\textbf{Actual Result} & Passed - accurately surmises whether the input coordinates are valid. \\ \hline
\textbf{Status} & \textcolor{green}{Passed} \\ \hline
\textbf{Priority} & High \\ \hline
\end{tabularx}

\subsection*{TC-003c}
\begin{tabularx}{\linewidth}{|l|L|}
\hline
\textbf{Test Case ID} & TC-003c \\ \hline
\textbf{Title} & The degToRad function accurately converts a number in degrees to radians. \\ \hline
\textbf{Requirement} & FR-002 \\ \hline
\textbf{Preconditions} & Input should be a number. \\ \hline
\textbf{Steps} & 
\begin{minipage}[t]{\linewidth}
\begin{enumerate}[leftmargin=1.4em,nosep,topsep=0pt]
  \item a number in degrees is input into the function.
  \item Function checks input for validity then returns the input value in radians. \vspace{1em}
\end{enumerate}
\end{minipage} \\ \hline
\textbf{Expected Result} & Returns the input number converted into radians if the input is valid, returns infinity otherwise. \\ \hline
\textbf{Actual Result} & Passed - accurately converts a valid input to radians and returns infinity otherwise. \\ \hline
\textbf{Status} & \textcolor{green}{Passed} \\ \hline
\textbf{Priority} & High \\ \hline
\end{tabularx}

\subsection*{TC-003d}
\begin{tabularx}{\linewidth}{|l|L|}
\hline
\textbf{Test Case ID} & TC-003c \\ \hline
\textbf{Title} & The createMap function loads a google map. (uses api mocking)\\ \hline
\textbf{Requirement} & FR-002 \\ \hline
\textbf{Preconditions} & The lat, lng, ref, and apikey inputs are valid. The lat and lng and tested for validness before the function call and the logic outside
the function affirms that apiKey and ref are valid. \\ \hline
\textbf{Steps} & 
\begin{minipage}[t]{\linewidth}
\begin{enumerate}[leftmargin=1.4em,nosep,topsep=0pt]
  \item The google maps api loader is initialized.
  \item The markers array input is filtered for valid markers or is set to an empty array if the markers input is bad.
  \item The google maps api loader is used to load the api.
  \item A new google map is created and valid markers are atached to the map. \vspace{1em}
\end{enumerate}
\end{minipage} \\ \hline
\textbf{Expected Result} & Creates a map with valid markers attached to it. \\ \hline
\textbf{Actual Result} & Passed - creates a map with valid markers attached to it, invalid markers are discarded and a bad markers input
is gracefully handled. \\ \hline
\textbf{Status} & \textcolor{green}{Passed} \\ \hline
\textbf{Priority} & High \\ \hline
\end{tabularx}

\subsection*{TC-004}
\textbf{Title:} Map Browse - Category Filter \\
\textbf{Requirement:} FR-002 \\
\textbf{Expected Result:} Only jobs of selected category are shown. \\
\textbf{Status:} Pending \quad \textbf{Priority:} Medium

\subsection*{TC-005}
\textbf{Title:} Place Bid - Valid \\
\textbf{Requirement:} FR-003 \\
\textbf{Preconditions:} Provider is logged in, KYC verified, job is open. \\
\textbf{Expected Result:} Bid document created; poster sees bid in Job Detail. \\
\textbf{Status:} Pending \quad \textbf{Priority:} High

\subsection*{TC-006}
\textbf{Title:} Accept Bid - Poster Flow \\
\textbf{Requirement:} FR-003 \\
\textbf{Expected Result:} Job status transitions to \texttt{awarded}; winning bid marked \texttt{accepted}. \\
\textbf{Status:} Pending \quad \textbf{Priority:} High

\subsection*{TC-007}
\textbf{Title:} Duo 2FA Challenge on Sensitive Action \\
\textbf{Requirement:} FR-005 \\
\textbf{Preconditions:} User logged in without recent 2FA; action = add payout method. \\
\textbf{Expected Result:} Duo prompt required; action proceeds only on success. \\
\textbf{Status:} Pending \quad \textbf{Priority:} Medium

\subsection*{TC-008a}
\begin{tabularx}{\linewidth}{|l|L|}
\hline
\textbf{Test Case ID} & TC-008a \\ \hline
\textbf{Title} & KYC Verification - Create Stripe Verification Session \\ \hline
\textbf{Requirement} & FR-004 \\ \hline
\textbf{Preconditions} & User is authenticated; KYC status = \texttt{pending}; real KYC mode enabled. \\ \hline
\textbf{Steps} & 
\begin{minipage}[t]{\linewidth}
\begin{enumerate}[leftmargin=1.4em,nosep,topsep=0pt]
  \item Send POST request to \texttt{/api/kyc/verification} with valid auth JWT token.
  \item System calls Stripe Identity API to create verification session.
  \item System stores session ID in user record.
  \item System returns Stripe verification URL and session ID.
\end{enumerate}
\end{minipage} \\ \hline
\textbf{Expected Result} & Returns Stripe verification URL (https://verify.stripe.com/*) and session ID (vs\_*) \\ \hline
\textbf{Actual Result} & Passed - Returns mocked Stripe URL and session ID. \\ \hline
\textbf{Status} & \textcolor{green}{Passed} \\ \hline
\textbf{Priority} & High \\ \hline
\end{tabularx}

\subsection*{TC-008b}
\begin{tabularx}{\linewidth}{|l|L|}
\hline
\textbf{Test Case ID} & TC-008b \\ \hline
\textbf{Title} & KYC Verification - Unauthorized Access \\ \hline
\textbf{Requirement} & FR-004, NFR-002 \\ \hline
\textbf{Preconditions} & User not authenticated or invalid token. \\ \hline
\textbf{Steps} & 
\begin{minipage}[t]{\linewidth}
\begin{enumerate}[leftmargin=1.4em,nosep,topsep=0pt]
  \item Send POST request to \texttt{/api/kyc/verification} without auth token or with invalid token.
\end{enumerate}
\end{minipage} \\ \hline
\textbf{Expected Result} & Returns 401 Unauthorized with error message. \\ \hline
\textbf{Actual Result} & Passed - Returns 401 with \texttt{\{error: "unauthorized"\}}. \\ \hline
\textbf{Status} & \textcolor{green}{Passed} \\ \hline
\textbf{Priority} & High \\ \hline
\end{tabularx}

\subsection*{TC-009a}
\begin{tabularx}{\linewidth}{|l|L|}
\hline
\textbf{Test Case ID} & TC-009a \\ \hline
\textbf{Title} & KYC Status Check \\ \hline
\textbf{Requirement} & FR-004 \\ \hline
\textbf{Preconditions} & User is authenticated. \\ \hline
\textbf{Steps} & 
\begin{minipage}[t]{\linewidth}
\begin{enumerate}[leftmargin=1.4em,nosep,topsep=0pt]
  \item Send GET request to \texttt{/api/kyc/status} with valid auth JWT token.
  \item System retrieves user KYC status from database.
  \item If verification session exists and status is \texttt{pending}, checks Stripe for updated status.
  \item System returns KYC status (\texttt{pending}, \texttt{verified}, or \texttt{failed}).
\end{enumerate}
\end{minipage} \\ \hline
\textbf{Expected Result} & Returns KYC status (pending, verified, or failed). \\ \hline
\textbf{Actual Result} & Passed - Returns valid status from database. \\ \hline
\textbf{Status} & \textcolor{green}{Passed} \\ \hline
\textbf{Priority} & High \\ \hline
\end{tabularx}

\subsection*{TC-009b}
\begin{tabularx}{\linewidth}{|l|L|}
\hline
\textbf{Test Case ID} & TC-009b \\ \hline
\textbf{Title} & KYC Status - Unauthorized Access \\ \hline
\textbf{Requirement} & FR-004, NFR-002 \\ \hline
\textbf{Preconditions} & User not authenticated. \\ \hline
\textbf{Steps} & 
\begin{minipage}[t]{\linewidth}
\begin{enumerate}[leftmargin=1.4em,nosep,topsep=0pt]
  \item Send GET request to \texttt{/api/kyc/status} without auth token.
\end{enumerate}
\end{minipage} \\ \hline
\textbf{Expected Result} & Returns 401 Unauthorized. \\ \hline
\textbf{Actual Result} & Passed - Returns 401 with error message. \\ \hline
\textbf{Status} & \textcolor{green}{Passed} \\ \hline
\textbf{Priority} & High \\ \hline
\end{tabularx}

\subsection*{TC-010}
\textbf{Title:} Messaging After Acceptance \\
\textbf{Requirement:} FR-006 \\
\textbf{Expected Result:} Parties can exchange messages in job thread; messages persist in Firestore. \\
\textbf{Status:} Planned \quad \textbf{Priority:} Medium

\subsection*{TC-011}
\textbf{Title:} Submit Review on Completion \\
\textbf{Requirement:} FR-007 \\
\textbf{Expected Result:} Review saved and visible on profile; duplicate review blocked. \\
\textbf{Status:} Planned \quad \textbf{Priority:} Medium

\subsection*{TC-012}
\textbf{Title:} Performance P95 - Map Query \\
\textbf{Requirement:} NFR-001 \\
\textbf{Expected Result:} P95 end-to-end from filter change to results render \(\leq 800\,\)ms on stub dataset. \\
\textbf{Status:} Planned \quad \textbf{Priority:} Low

\subsection*{TC-013}
\textbf{Title:} Ruleset Audit - No Write Without Claims \\
\textbf{Requirement:} NFR-002 \\
\textbf{Expected Result:} Firestore rules reject writes missing \texttt{kycVerified == true}. \\
\textbf{Status:} Pending \quad \textbf{Priority:} High

\subsection*{TC-014}
\textbf{Title:} Accessibility - Keyboard Nav on Map/List \\
\textbf{Requirement:} NFR-003 \\
\textbf{Expected Result:} Tabbing reaches filters, list items, and primary actions; visible focus outlines present. \\
\textbf{Status:} Planned \quad \textbf{Priority:} Low

\subsection*{TC-015a}
\begin{tabularx}{\linewidth}{|l|L|}
\hline
\textbf{Test Case ID} & TC-015a \\ \hline
\textbf{Title} & Profile Page - Display KYC Pending Status \\ \hline
\textbf{Requirement} & FR-004 \\ \hline
\textbf{Preconditions} & User authenticated, KYC status = \texttt{pending}. \\ \hline
\textbf{Steps} & 
\begin{minipage}[t]{\linewidth}
\begin{enumerate}[leftmargin=1.4em,nosep,topsep=0pt]
  \item Navigate to \texttt{/profile}.
  \item Profile component renders and displays account status section.
  \item System retrieves user KYC status from session.
\end{enumerate}
\end{minipage} \\ \hline
\textbf{Expected Result} & KYC status displays "Pending" tag with appropriate styling. \\ \hline
\textbf{Actual Result} & Passed - Profile page correctly displays pending status. \\ \hline
\textbf{Status} & \textcolor{green}{Passed} \\ \hline
\textbf{Priority} & High \\ \hline
\end{tabularx}

\subsection*{TC-015b}
\begin{tabularx}{\linewidth}{|l|L|}
\hline
\textbf{Test Case ID} & TC-015b \\ \hline
\textbf{Title} & Profile Page - Display KYC Verified Status \\ \hline
\textbf{Requirement} & FR-004 \\ \hline
\textbf{Preconditions} & User authenticated, KYC status = \texttt{verified}. \\ \hline
\textbf{Steps} & 
\begin{minipage}[t]{\linewidth}
\begin{enumerate}[leftmargin=1.4em,nosep,topsep=0pt]
  \item Navigate to \texttt{/profile}.
  \item Profile component renders with verified KYC status.
\end{enumerate}
\end{minipage} \\ \hline
\textbf{Expected Result} & KYC status displays "Verified" tag; no action buttons shown. \\ \hline
\textbf{Actual Result} & Passed - Profile page displays verified status correctly. \\ \hline
\textbf{Status} & \textcolor{green}{Passed} \\ \hline
\textbf{Priority} & High \\ \hline
\end{tabularx}

\subsection*{TC-015c}
\begin{tabularx}{\linewidth}{|l|L|}
\hline
\textbf{Test Case ID} & TC-015c \\ \hline
\textbf{Title} & Profile Page - Complete KYC Button Displayed \\ \hline
\textbf{Requirement} & FR-004 \\ \hline
\textbf{Preconditions} & User authenticated, KYC status = \texttt{pending}. \\ \hline
\textbf{Steps} & 
\begin{minipage}[t]{\linewidth}
\begin{enumerate}[leftmargin=1.4em,nosep,topsep=0pt]
  \item Navigate to \texttt{/profile}.
  \item System checks KYC status is not verified.
  \item Profile component renders action buttons.
\end{enumerate}
\end{minipage} \\ \hline
\textbf{Expected Result} & "Complete KYC" and "Refresh Status" buttons are visible. \\ \hline
\textbf{Actual Result} & Passed - Action buttons displayed for pending status. \\ \hline
\textbf{Status} & \textcolor{green}{Passed} \\ \hline
\textbf{Priority} & High \\ \hline
\end{tabularx}

\subsection*{TC-015d}
\begin{tabularx}{\linewidth}{|l|L|}
\hline
\textbf{Test Case ID} & TC-015d \\ \hline
\textbf{Title} & Profile Page - Initiate KYC Verification \\ \hline
\textbf{Requirement} & FR-004 \\ \hline
\textbf{Preconditions} & User authenticated, KYC pending, production mode. \\ \hline
\textbf{Steps} & 
\begin{minipage}[t]{\linewidth}
\begin{enumerate}[leftmargin=1.4em,nosep,topsep=0pt]
  \item User clicks "Complete KYC" button.
  \item System calls \texttt{/api/kyc/verification} endpoint.
  \item System receives Stripe verification URL.
  \item System opens URL in new browser tab.
\end{enumerate}
\end{minipage} \\ \hline
\textbf{Expected Result} & Stripe Identity verification opens in new tab; notification displayed. \\ \hline
\textbf{Actual Result} & Passed - Verification URL opened correctly. \\ \hline
\textbf{Status} & \textcolor{green}{Passed} \\ \hline
\textbf{Priority} & High \\ \hline
\end{tabularx}

\subsection*{TC-015e}
\begin{tabularx}{\linewidth}{|l|L|}
\hline
\textbf{Test Case ID} & TC-015e \\ \hline
\textbf{Title} & Profile Page - Refresh KYC Status \\ \hline
\textbf{Requirement} & FR-004 \\ \hline
\textbf{Preconditions} & User authenticated, KYC pending. \\ \hline
\textbf{Steps} & 
\begin{minipage}[t]{\linewidth}
\begin{enumerate}[leftmargin=1.4em,nosep,topsep=0pt]
  \item User clicks "Refresh Status" button.
  \item System calls \texttt{/api/kyc/status} endpoint.
  \item System receives updated status.
  \item System updates UI and session with new status.
\end{enumerate}
\end{minipage} \\ \hline
\textbf{Expected Result} & KYC status refreshed; UI updated with current status. \\ \hline
\textbf{Actual Result} & Passed - Status refresh works correctly. \\ \hline
\textbf{Status} & \textcolor{green}{Passed} \\ \hline
\textbf{Priority} & High \\ \hline
\end{tabularx}

\subsection*{TC-016a}
\begin{tabularx}{\linewidth}{|l|L|}
\hline
\textbf{Test Case ID} & TC-016a\\ \hline
\textbf{Title} & Profile Page - Upload Avatar Successfully \\ \hline
\textbf{Requirement} & FR-008 (User Profile Management) \\ \hline
\textbf{Preconditions} & User authenticated. \\ \hline
\textbf{Steps} & 
\begin{minipage}[t]{\linewidth}
\begin{enumerate}[leftmargin=1.4em,nosep,topsep=0pt]
  \item User clicks "Change Avatar" button.
  \item User selects valid image file (PNG, < 5MB).
  \item System uploads file to storage.
  \item System updates user profile with avatar URL.
\end{enumerate}
\end{minipage} \\ \hline
\textbf{Expected Result} & Avatar uploaded; preview displayed; success notification shown. \\ \hline
\textbf{Actual Result} & Passed - Avatar upload successful. \\ \hline
\textbf{Status} & \textcolor{green}{Passed} \\ \hline
\textbf{Priority} & Medium \\ \hline
\end{tabularx}

\subsection*{TC-016b}
\begin{tabularx}{\linewidth}{|l|L|}
\hline
\textbf{Test Case ID} & TC-016b \\ \hline
\textbf{Title} & Profile Page - Reject Large Avatar Files \\ \hline
\textbf{Requirement} & FR-008, NFR-002 (Security) \\ \hline
\textbf{Preconditions} & User authenticated. \\ \hline
\textbf{Steps} & 
\begin{minipage}[t]{\linewidth}
\begin{enumerate}[leftmargin=1.4em,nosep,topsep=0pt]
  \item User clicks "Change Avatar" button.
  \item User selects image file larger than 5MB.
  \item System validates file size before upload.
\end{enumerate}
\end{minipage} \\ \hline
\textbf{Expected Result} & Upload rejected; error message "Image must be smaller than 5MB" displayed. \\ \hline
\textbf{Actual Result} & Passed - Large files rejected correctly. \\ \hline
\textbf{Status} & \textcolor{green}{Passed} \\ \hline
\textbf{Priority} & Medium \\ \hline
\end{tabularx}

\subsection*{TC-016c}
\begin{tabularx}{\linewidth}{|l|L|}
\hline
\textbf{Test Case ID} & TC-016c \\ \hline
\textbf{Title} & Profile Page - Handle Avatar Upload Error \\ \hline
\textbf{Requirement} & FR-008 \\ \hline
\textbf{Preconditions} & User authenticated; network/server error occurs. \\ \hline
\textbf{Steps} & 
\begin{minipage}[t]{\linewidth}
\begin{enumerate}[leftmargin=1.4em,nosep,topsep=0pt]
  \item User selects valid image file.
  \item System attempts upload to \texttt{/api/avatar/upload}.
  \item Server returns error response.
\end{enumerate}
\end{minipage} \\ \hline
\textbf{Expected Result} & User-friendly error message displayed; no profile update. \\ \hline
\textbf{Actual Result} & Passed - Error handled gracefully. \\ \hline
\textbf{Status} & \textcolor{green}{Passed} \\ \hline
\textbf{Priority} & Medium \\ \hline
\end{tabularx}

\subsection*{TC-016d}
\begin{tabularx}{\linewidth}{|l|L|}
\hline
\textbf{Test Case ID} & TC-016d \\ \hline
\textbf{Title} & Profile Page - Display Existing Avatar \\ \hline
\textbf{Requirement} & FR-008 \\ \hline
\textbf{Preconditions} & User authenticated; avatar URL exists in profile. \\ \hline
\textbf{Steps} & 
\begin{minipage}[t]{\linewidth}
\begin{enumerate}[leftmargin=1.4em,nosep,topsep=0pt]
  \item Navigate to \texttt{/profile}.
  \item System retrieves user avatar URL from database.
  \item Profile component renders avatar image.
\end{enumerate}
\end{minipage} \\ \hline
\textbf{Expected Result} & User's avatar displayed in profile header. \\ \hline
\textbf{Actual Result} & Passed - Existing avatar displayed correctly. \\ \hline
\textbf{Status} & \textcolor{green}{Passed} \\ \hline
\textbf{Priority} & Low \\ \hline
\end{tabularx}

% ===================== 5. Code Repository and Branching Strategy =====================
\section{Code Repository and Branching Strategy}
\textbf{Repository:} \url{https://github.com/your-org/openbid} \textit{(placeholder)} \\[4pt]
\textbf{Branches:}
\begin{itemize}[leftmargin=1.4em]
  \item \texttt{main}: stable releases.
  \item \texttt{develop}: ongoing integration.
  \item \texttt{feature/\{slug\}}: e.g., \texttt{feature/auth}, \texttt{feature/map}, \texttt{feature/bids}, \texttt{feature/kyc}, \texttt{feature/tests}.
\end{itemize}

% ===================== 6. Task Allocation and Timeline =====================
\section{Task Allocation and Timeline}
\textbf{Pair-Programming Rotation (weekly):} two pairs; driver/navigator swap daily; Scrum Master rotates weekly (Tyler \(\rightarrow\) Mani \(\rightarrow\) Yanness \(\rightarrow\) Alaister).

\subsection*{Sprint Plan (3 weeks for Iteration 3)}
\begin{itemize}[leftmargin=1.4em]
  \item \textbf{Week 1:} Implement auth + Duo 2FA hook, KYC gate stub, Post Job UI/API, initial tests (TC-001, TC-002, TC-007, TC-008).
  \item \textbf{Week 2:} Map browse (pins, filters), Place Bid flow, tests (TC-003, TC-004, TC-005, TC-006, TC-012).
  \item \textbf{Week 3:} Prototype hardening, accessibility pass (keyboard focus), performance harness scaffolding, test documentation (TC-011, TC-013 planning).
\end{itemize}

\subsection*{Task Breakdown (examples)}
\begin{itemize}[leftmargin=1.4em]
  \item Frontend: NewJob form, MapView, JobList, Bid modal, validation.
  \item Backend (Express on Firebase Functions/Cloud Run): endpoints for jobs, bids; KYC/Duo middleware.
  \item Firestore rules: enforce \texttt{kycVerified == true} for job/bid writes.
  \item Testing: Vitest/Jest + React Testing Library for UI; supertest for API; rules-unit-testing for Firestore.
  \item Docs: update traceability, add test run notes.
\end{itemize}

% ===================== 7. Prototype Overview =====================
\section{Prototype Overview}
\textbf{Stack (MVP):} React + SCSS (Firebase Hosting), Node.js + Express (Firebase Functions or Cloud Run), Firestore, Google Maps JS, Stripe (Connect/Identity), Duo 2FA.

\subsection*{Implemented in Iteration 3 (target)}
\begin{itemize}[leftmargin=1.4em]
  \item Auth shell with Duo challenge on sensitive action.
  \item KYC gate (UI + rules) using a sandbox token flow.
  \item Post Job UI + server write; basic Job Detail.
  \item Map browse with radius/category filters on seeded data.
  \item Place Bid basic happy path.
\end{itemize}

% ===================== 8. Submission Guidelines =====================
\section{Submission Guidelines}
\begin{itemize}[leftmargin=1.4em]
  \item \textbf{GitHub:} Push code and documentation; tag release as \texttt{ITR2.1}.
  \item \textbf{eClass PDF:} include \textit{Prototype Overview}, \textit{Updated Requirements/Use Cases}, \textit{Test Plan and (initial) Results}, and \textit{Updated Iteration Plan/Backlog}.
\end{itemize}

% ===================== Appendix A: Sample Test Skeletons =====================
\section*{Appendix A: Test Skeletons (illustrative)}
\subsection*{Frontend (React Testing Library)}
\begin{lstlisting}
import { render, screen, fireEvent } from '@testing-library/react'
import NewJob from '../src/pages/NewJob'

test('TC-002: shows validation error when title empty', async () => {
  render(<NewJob />)
  fireEvent.click(screen.getByRole('button', { name: /post/i }))
  expect(await screen.findByText(/title is required/i)).toBeInTheDocument()
})
\end{lstlisting}

\subsection*{API (Express + supertest)}
\begin{lstlisting}
import request from 'supertest'
import app from '../functions/app'

test('TC-007: blocks job create when kyc not verified', async () => {
  const token = await getAuthToken({ kycVerified: false })
  const res = await request(app)
    .post('/jobs')
    .set('Authorization', `Bearer ${token}`)
    .send({ title: 'Yard help', ... })
  expect(res.status).toBe(403)
})
\end{lstlisting}

\subsection*{Firestore Rules (rules-unit-testing)}
\begin{lstlisting}
it('TC-012: rejects write without kycVerified', async () => {
  const db = authedDB({ uid: 'u1', kycVerified: false })
  const ref = db.collection('jobs').doc('j1')
  await assertFails(ref.set({ title: 'T', ... }))
})
\end{lstlisting}

\subsection*{KYC Routes (Express + supertest + Jest)}
\begin{lstlisting}
import request from 'supertest'
import express from 'express'

// TC-008a: Create Stripe verification session
test('should return Stripe verification URL', async () => {
  const response = await request(app)
    .post('/api/kyc/verification')
    .set('Authorization', 'Bearer fake-token')
    .expect(200)

  expect(response.body.url).toContain('https://verify.stripe.com/')
  expect(response.body.sessionId).toContain('vs')
})

// TC-009a: Check KYC status
test('should return KYC status', async () => {
  const response = await request(app)
    .get('/api/kyc/status')
    .set('Authorization', 'Bearer fake-token')
    .expect(200)

  expect(['verified', 'pending', 'failed'])
    .toContain(response.body.status)
})

// TC-008b: Unauthorized access
test('should reject unauthorized user', async () => {
  mockAuth.auth.verify.mockResolvedValue(null)
  
  const response = await request(app)
    .post('/api/kyc/verification')
    .expect(401)

  expect(response.body.error).toBe('unauthorized')
})
\end{lstlisting}

\end{document}
